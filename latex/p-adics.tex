\label{sec:p-adics}
\section{The $p$-adics}

The $p$-adics $\Q_p$ are an alternate completion of the rationals $\Q$, with respect to the $p$-adic absolute value.
\footnote{The standard completion of $\Q$, of course, is $\R$ with respect
to the familiar absolute value} They will be useful in our discussion of solving ECDLP for curves of trace one.
\todo{Introduce what p-adics look like here?}
\begin{defn}
For a rational number $a$ and a prime number $p$, separate out all factors of $p$ from $a$ and write: $$ a = p^r \dfrac{m}{n} $$ where $r$, $m$ and $n$ are integers, and $p$ does not divide $m$ or $n$. The exponent $r$ is called the $\textbf{p-adic ordinal}$ of $a$, denoted $\text{ord}_p(a)$.
\end{defn}

% The absolute value
\begin{defn}
For a prime $p$, we define a function $|.|_p : \Q \to \Q_{\ge 0}$ where for $a \in \Q$:
$$
|a|_p = \left\{
        \begin{array}{ll}
            p^{-\text{ord}_p(a)} & \quad a \neq 0 \\
            0 & \quad a = 0.
        \end{array}
    \right.
$$
The function $|.|_p$ is called the $\textbf{p-adic absolute value}$.
\end{defn}

\begin{prop}
The p-adic absolute value is a norm on $\Q$, and induces a metric $$d_p(a, \ b) = | a - b |_p$$ for $a, b \in \Q$.
\end{prop}

\begin{pf}
\todo{Prove this!}
\end{pf}

\todo[inline, caption="p-adic comment"]{Explain why we don't really detailed proofs of Cauchy sequences etc}

\begin{defn}
A p-adic number $a$ is called a $\textbf{p-adic integer}$ if $ord_p(a) \ge 0$. The set of all p-adic integers is
denoted $\Z_p$.
\end{defn}

\begin{rmk}
A p-adic integer is always of the form $$a_0 + a_1p + a_2p^2 + ... ,$$ i.e., all powers of $p$ are non-negative.
\end{rmk}

\subsection{Computing lifts and reducing modulo $p$}
Since $\Q_p$ is a field with characteristic 0, we can talk about elliptic curves over the $p$-adics, and all of the
theory we built up in the previous sections applies.

\todo{Describe computation of lifts}

Going the other way is simple. We define a map from $E(\Q_p)$ to $\tilde{E}(\F_p)$
by reducing a point modulo $p$, i.e, extracting its $a_0$ term.
\todo{more on reduction mod p}

\subsection{More about elliptic curves over $\Q_p$}

\begin{defn}
Let $E(\Q_p)$ be an elliptic curve. The group $E_1(\Q_p)$ is defined to be:
$$ E_1(\Q_p) = \{ P \in E(\Q_p) \ | \ \tilde{P} = \OO  \}. $$
In words, $E_1$ is the set of points on $E$ that reduce modulo $p$ to $\OO$.
This leads naturally to the following proposition:
\end{defn}

\begin{prop}
$$E(\Q_p) / E_1(\Q_p) \simeq E(\F_p).$$
\end{prop}

\begin{pf}
Define a map $r: E(\Q_p) \to E(F_p)$ where $r(P) = \tilde{P}$. \todo{Prove this is a homomorphism} By definition,
the kernel of this map is $E_1(\Q_p)$. The result follows from the First Isomorphism Theorem.
\end{pf}

\begin{defn}
The subgroup $E_n$ (for $n \in N$) of $E(\Q_p)$  is defined:
$$ E_n(\Q_p) = \{ P \in E(\Q_p) \ | \ \mathrm{ord}_p(x_P) \le -2n \} \cup \{ \OO \}, $$
where $x_P$ is the x-coordinate of $P$.
\end{defn}


