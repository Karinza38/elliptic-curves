\chapter{Elliptic Curve Basics}%
\label{sec:elliptic-curves}

% Weierstrass equation
\begin{defn}
Let $K$ be a field. An $\textbf{elliptic curve $E$ over $K$}$ is defined by an equation:
$$E : y^2 + a_1xy + a_3y = x^3 + a_2x^2 + a_4x + a_6$$
where $a_1, a_2, a_3, a_4, a_6 \in K$ and the $\textbf{discriminant}$ $\Delta$ is non-zero. This equation is called a $\textbf{Weierstrass equation}$.
\end{defn}

% L-rational points
\begin{defn}
With $K$ and $E$ defined as above, the set of $\textbf{L-rational points}$ on $E$ for any extension $L$ of $K$ is the set of pairs $(x, y) \in L \times L$ that
satisfy $E$, together with $\OO$, the point at infinity.

The set of L-rational points is denoted $E(L)$.
\end{defn}

% Trace of Frobenius
\begin{defn}
Let $E$ be an elliptic curve over a finite field $\finfield$. The $\textbf{trace of Frobenius}$ t is defined by:
$$ \#E(\finfield) = q + 1 - t, $$
where $\#E(\finfield)$ is the number of elements in $E(\finfield)$.
\end{defn}

\begin{rmk}
The trace of Frobenius is equal to one if and only if $E(\finfield)$ has exactly $q$ elements. This has important implications
for cryptography, as we will see.
\end{rmk}
